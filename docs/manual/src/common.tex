\begin{titlepage}
  \begin{center}

  {\Huge AXIS SPI MASTER}

  \vspace{25mm}

  \includegraphics[width=0.90\textwidth,height=\textheight,keepaspectratio]{img/AFRL.png}

  \vspace{25mm}

  \today

  \vspace{15mm}

  {\Large Jay Convertino}

  \end{center}
\end{titlepage}

\tableofcontents

\newpage

\section{Usage}

\subsection{Introduction}

\par
The intent of this core is to provide a base AXIS to SPI Master interface. It is capable of back to back transfers with zero wait time.
The data can be output at any rate up to half the input clock. The core SPI clock is generated for external use only and should
NOT be routed into any logic. This device also does the chip selection based on the current activity of the core. CPOL/CPHA can
be altered at anytime.

\subsection{Dependencies}

\par
The following are the dependencies of the cores.

\begin{itemize}
  \item fusesoc 2.X
  \item iverilog (simulation)
  \item cocotb (simulation)
\end{itemize}

\input{src/fusesoc/depend_fusesoc_info.tex}

\subsection{In a Project}
\par
This core connects a SPI to the AXIS bus. Meaning this is a streaming device only. Connect the MOSI/MISO to the SPI device in question and connect the AXIS to its intended endpoints.

\section{Architecture}
\par
The core for this contains the following:
\begin{itemize}
  \item \textbf{axis\_spi} Interface with SPI to AXIS interface.
  \item \textbf{mod\_clock\_ena\_gen} Generate an enable used to sample data for piso/sipo.
  \item \textbf{piso} Take parallel data and output it in serial.
  \item \textbf{sipo} Take serial data and output it in parallel.
\end{itemize}

The main core is made to interface a AXIS bus to the SPI bus. This is done using the SIPO and PISO cores to change from serial to parallel data streams.
In addition mod clock enable gen cores create the negative and positive enables based upon the input clock and set rate to sample the data. This is then
glued together in the core with some logic to output the approtpriate SPI signals, including a generated clock. This generated clock is created by
the mod clock gen enables only and is NOT used to clock any internal signals. Use only as a output clock. The core allows for any rate to be used up to
half the input clock. The clock phase and polarity can be changed on the fly at anytime. All word transfers are the size of the AXIS bus. If multiple
byte transfers of varing sizes are needed. It is recommened to set this to one byte width for the AXIS data bus and do back to back transfers for the
number needed. Basically having data available to the core as soon as it can get it means there will be no gap in the spi output.

\subsection{Ports}
A port list is availabe, with specific signal information, in the \ref{Module Documentation}. The generalized idea is a AXIS slave input
for MOSI data. A AXIS master output for MISO data, and a master SPI interface with sclk, mosi, miso, and ss\_n signals. The dcount outputs
give insight into the status of the in/out data bits of the core.

\subsection{Waveforms}
The idealized simulation waveforms are shown below. The values reflect the results of using the icarus backend with GTKwaveform view tools.

\par
Back to back transfers will naturally occur if data is available in time for the next ready. This allows for zero intertransmission gaps.
\begin{figure}[H]
\caption{CPOL = 0 : CPHA = 0}
\centering
\includegraphics[width=\textwidth]{img/diagrams/waveform_00_back_to_back.png}
\end{figure}

\par
The following figures are not back to back transfers. Meaning there is a some gap in the time a new word is available. This could be done on purpose
due to SPI slave needs and can easily done by looking at the counter values of the core.
\begin{figure}[H]
\caption{CPOL = 0 : CPHA = 0}
\centering
\includegraphics[width=\textwidth]{img/diagrams/waveform_00_spaced.png}
\end{figure}

\begin{figure}[H]
\caption{CPOL = 0 : CPHA = 1}
\centering
\includegraphics[width=\textwidth]{img/diagrams/waveform_01_spaced.png}
\end{figure}

\begin{figure}[H]
\caption{CPOL = 1 : CPHA = 0}
\centering
\includegraphics[width=\textwidth]{img/diagrams/waveform_10_spaced.png}
\end{figure}

\begin{figure}[H]
\caption{CPOL = 1 : CPHA = 1}
\centering
\includegraphics[width=\textwidth]{img/diagrams/waveform_11_spaced.png}
\end{figure}

\section{Building}

\par
The AXIS SPI is written in Verilog 2001. It should synthesize in any modern FPGA software. The core comes as a fusesoc packaged core and can be
included in any other core. Be sure to make sure you have meet the dependencies listed in the previous section.

\subsection{fusesoc}
\par
Fusesoc is a system for building FPGA software without relying on the internal project management of the tool. Avoiding vendor lock in to Vivado or Quartus.
These cores, when included in a project, can be easily integrated and targets created based upon the end developer needs. The core by itself is not a part of
a system and should be integrated into a fusesoc based system. Simulations are setup to use fusesoc and are a part of its targets.

\subsection{Source Files}

\input{src/fusesoc/files_fusesoc_info.tex}

\subsection{Targets}

\subsubsection{fusesoc\_info Targets}
\begin{itemize}
\item default
	\begin{itemize}
	\item[$\space$] Info: Default for IP intergration.
	\end{itemize}
\item lint
	\begin{itemize}
	\item[$\space$] Info: Lint with Verible
	\end{itemize}
\item sim
	\begin{itemize}
	\item[$\space$] Info: Base simulation using icarus as default.
	\end{itemize}
\item sim\_rand\_data
	\begin{itemize}
	\item[$\space$] Info: Use random data as sim input.
	\end{itemize}
\item sim\_8bit\_count\_data
	\begin{itemize}
	\item[$\space$] Info: Use counter data as sim input.
	\end{itemize}
\item sim\_cocotb
	\begin{itemize}
	\item[$\space$] Info: Cocotb unit tests
	\end{itemize}
\end{itemize}


\subsection{Directory Guide}

\par
Below highlights important folders from the root of the directory.

\begin{enumerate}
  \item \textbf{docs} Contains all documentation related to this project.
    \begin{itemize}
      \item \textbf{manual} Contains user manual and github page that are generated from the latex sources.
    \end{itemize}
  \item \textbf{src} Contains source files for the core
  \item \textbf{tb} Contains test bench files for iverilog and cocotb
    \begin{itemize}
      \item \textbf{cocotb} testbench files
    \end{itemize}
\end{enumerate}

\newpage

\section{Simulation}
\par
There are a few different simulations that can be run for this core.

\subsection{iverilog}
\par
iverilog is used for simple test benches for quick verification, visually, of the core.
\begin{itemize}
  \item \textbf{sim} Standard simulation of SPI looped, input/output verification.
\end{itemize}

This uses a axis stimulator cores for master/slave. This will run all the data in the slave axis SPI interface, which
will output the data over the SPI interface. This is then looped into the SPI input that then puts the valid data out
on the master axis SPI interface.

\subsection{cocotb}
\par
To use the cocotb tests you must install the following python libraries.
\begin{lstlisting}[language=bash]
  $ pip install cocotb
  $ pip install cocotbext-axi
  $ pip install cocotbext-spi
\end{lstlisting}

Each module has a cocotb based simulation. These use the cocotb extensions made by Alex.
The two extensions used are cocotbext-axi and cocotbext-spi. These provide outside verification
of the implimentation. These tests consist of the following fusesoc targets.

\begin{itemize}
  \item \textbf{sim\_cocotb} Standard simulation of SPI data to and from cocotbexts this tests all CPOL/CPHA options.
\end{itemize}

Then you must use the cocotb sim target. The targets above can be run with the following:
\begin{lstlisting}[language=bash]
  $ fusesoc run --target sim_cocotb AFRL:device_converter:axis_spi:1.0.0
\end{lstlisting}

\newpage

\section{Module Documentation} \label{Module Documentation}

\begin{itemize}
  \item \textbf{axis\_spi} Interfaces AXIS to SPI.\\
  \item \textbf{tb\_spi} Verilog test bench.\\
  \item \textbf{tb\_cocotb verilog} Verilog test bench base for cocotb.\\
  \item \textbf{tb\_cocotb python} cocotb unit test functions.\\
\end{itemize}


